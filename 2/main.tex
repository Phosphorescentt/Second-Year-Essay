\section{The one dimensional case}
To develop an understanding of how to analyse these networks, we shall consider the one dimensional case where our network is the lattice on $\mathbb{Z}$, or a "chain". % We shall use the same
% idea of percolation probability from definitions \ref{def:site percolation} and \ref{def:bond percolation}.

\begin{theorem}\label{thm:critical probability z1 site percolation}
  The critical probability, $p_c$, for the lattice on $\mathbb{Z}^2$ when considering site percolation is $p_c = 1$.
\end{theorem}

If we recall the definition of the critical probability, the value of $p_c$ that we're looking for is intuitively $1$. In order to prove this theorem, we shall introduce some more
machinery. This machinery isn't necessary for the proof, but it helps with understanding and will allow us to analyse more interesting cases later on.

\begin{definition}\label{def:s-clusters per site}
  If the percolation probability on the chain is $p$, we define the \textbf{number of $\mathbf{s}$-clusters per site} by the following quantity:
  $$n_s = p^s(1-p)^2$$
\end{definition}

This quantity represents the probability that any given site in the network is the left end of an $s$-cluster. So, if our network is of length $L >> s$, then we will have
$Ln_s=Lp^s(1-p)^2$ clusters of size $s$ on average. This quantity also allows us to explore the probability that any given site is part of an $s$-cluster. Such a probability is
given by the quantity $n_ss$.

The idea of percolation is to understand whether a path exists from one side of the network to the other, so we have to use a slightly different definition for the one dimensional
case. This new definition fixes an issue that I'll address after the proof.

\begin{definition}\label{def:critical probability special case}
  Consider the lattice on $\mathbb{Z}$ of the form $N = (V, E)$. The critical probability, denoted $p_c$, is the percolation probability such that the probability that there
  \textbf{uniquely} exists a cluster, $C
  \subseteq V$, with $|C| = \infty$ is $1$.
\end{definition}

This proof and the following corollory are heavily inspired by the proof and subsequent corollory from Dietrich Stauffer's \textit{Introduction to Percolation Theory}. \cite{Stauffer}

\begin{proof}{(Theorem \ref{thm:critical probability z1 site percolation})}
  We shall prove by contradiction. Let us assume that $p_c \in [0, 1)$ and is fixed. This implies that a chain of length $L$ will have, on average, $L(1-p)$ closed sites. As $L
  \rightarrow \infty$, $L(1-p) \rightarrow \infty$ showing us that there is at least one closed site in the chain and that means there is no continuous row of occupied
  sites. Thus $p_c=1$ in order to have only one infinite cluster.
\end{proof}

So why doesn't this proof work if we hadn't made that ammendment to the definition. Notice that $\mathbb{Z}$ is a countably infinite set and all of the $L(1-p)$ closed sites form a
subset of $\mathbb{Z}$. This means that the set of all closed sites is also countably infinite. This situation may be rephrased in a way such that we must partition a countably infinite set into
countably infinite subsets. \todo{prove this?}

The above results allow us to get some more interesting information about the behaviour of our system. For example, we can deduce the following:
\begin{corollary}
  When considering the lattice on $\mathbb{Z}$ with percolation probability $p \in [0, 1)$, the following equality holds:
  $$\sum_sn_ss = p$$
\end{corollary}
This result comes from the fact that every open site must belong to a cluster of some size $s$. So summing $n_ss$ over all $s$ must give us $p$. This equality may also be derived
using the definition of $n_s$ and the formula for a geometric series.

\begin{proof}
  \begin{align*}
    \sum_sn_ss &= \sum_sp^s(1-p)^2s \\
    &= (1-p)^2\sum_sp\frac{d(p^s)}{dp} \\
    &= (1-p)^2p\frac{d(\sum_sp^s)}{dp} \\
    &= (1-p)^2p\frac{d(p/(1-p))}{dp} \\
    &= p
  \end{align*}
\end{proof}

It's worth noting that this equality doesn't hold for $p=1$, because $n_s=1^s(1-1)^2=0$ so $\sum_sn_ss = 0$. The technique of considering the sizes of clusters and the number of
empty sites surrounding them is also used when analysing more compelx structures. Say, for example, you have a cluster, $C$, of size $|C|=9$ on the lattice on $\mathbb{Z}^2$. This
cluster could take many different shapes with different numbers of empty sites surrounding it. For example, this cluster could be a straight line of open sites which has a total of
$22$ closed sites surrounding it. Therefore the probability that a cluster like this exists at any site given a percolation probability, $p$, is $p^{9}(1-p)^{20}$. This cluster could also be a
square of open sites with side length $3$. This would have $12$ closed sites surrounding it and thus the probability of a cluster of this shape existing at any given site is
$p^{9}(1-p)^{12}$. The ability for clusters to be the same size but have different "perimeters" is what makes analysing these problems in higher dimensions so difficult.
