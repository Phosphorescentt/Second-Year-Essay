\section{Analysis of different network configurations}
In this section, we will look at a couple of techniques used to analyse different network configurations to determine their critical probabilities.

\subsection{The one dimensional case}
To develop an understanding of how to analyse these networks, we shall consider the one dimensional case where our network is the lattice on $\mathbb{Z}$, or a "chain".
If we recall the definition of the critical probability, the value of $p_c$ that we're looking for is intuitively $1$ because that's the only way that everything is connected. This
is true for both site and bond percolation. In
order to more rigorously understand this result, we shall introduce some more
machinery. This machinery isn't necessary for the "proof", but it helps with understanding and will allow us to analyse more interesting cases later on. The following line of
argument is inspired by Dietrich Stauffer's \textit{Introduction to Percolation Theory} \cite{Stauffer}.

If we let each of the sites in our chain be open with probability $p$, then the probability that any $s$ arbitrary sites are all open is $p^s$. Now if we wanted to create a cluster
in our chain of size $s$, we would have to have one closed site followed by $s$ open sites followed by a final closed site to cap it off. This means that the probability of any
given site being the left end of a cluster of size $s$ is $(1-p)p^s(1-p) = p^s(1-p)^2$.

So now how many clusters of size $s$ are there in our chain? If we suppose that the length of the chain is $L >> s$, then we have $Lp^s(1-p)^2$ clusters of size $s$ in our chain of
length $L$ (ignoring small errors produced at the right end of the chain). It's practical for us to talk about the number of clusters at any given site in the chain which is
$\frac{Lp^s(1-p)^2}{L} = p^s(1-p)^2$ which we denote by $n_s$ and refer to as the \textbf{normalised cluster number}. Now it's clear that the probability of any site belonging to a
cluster of size $s$ is larger than our normalised cluster number by a factor of $s$ yielding $n_ss$. Now we can consider the critical probability for this configuration. We want
every site in the chain to be occupied so that there are no holes. If $p = 1$, this is clearly true and the whole chain is one single cluster. For $p < 1$ there will be some holes.
For a chain of length $L$, on average we will have $L(1-p)$ empty sites. As we let $L \rightarrow \infty$ for a fixed $p < 1$ we get that $L(1-p) \rightarrow \infty$. Therefore
there is at least one empty site in the chain so we don't have percolation. Thus $p_c = 1$.

The above results allow us to get some more interesting information about the behaviour of our system. For example, we can deduce the following:
\begin{corollary}
  When considering the lattice on $\mathbb{Z}$ with percolation probability $p \in [0, 1)$, the following equality holds:
  $$\sum_sn_ss = p$$
\end{corollary}
This result comes from the fact that every open site must belong to a cluster of some size $s$. So summing $n_ss$ over all $s$ must give us $p$. This equality may also be derived
using the definition of $n_s$ and the formula for a geometric series.

\begin{proof}
  \begin{align*}
    \sum_sn_ss &= \sum_sp^s(1-p)^2s \\
    &= (1-p)^2\sum_sp\frac{d(p^s)}{dp} \\
    &= (1-p)^2p\frac{d(\sum_sp^s)}{dp} \\
    &= (1-p)^2p\frac{d(p/(1-p))}{dp} \\
    &= p
  \end{align*}
\end{proof}

It's worth noting that this equality doesn't hold for $p=1$, because $n_s=1^s(1-1)^2=0$ so $\sum_sn_ss = 0$.

\subsection{Bethe Lattice}
Another easy case to analyse is the Bethe lattice as shown in Figure \ref{fig:bethe lattice}. The Bethe lattice is another configuration that's easy for us to analyse.

When looking at the infinite Bethe lattice, it's clear that we could choose any site to be the root of the lattice just by moving all the other sites around to make the visual
representation clearer. This self-similarity means that we only have to consider one branch of the Bethe lattice and if that percolates then so does the whole structure. So if we
assume that some part of the lattice has already been percolated through and look at what's remaining we can see that we're left with a binary tree. In order for percolation to
take place here, we need that at least one of these branches be present so we require that $p \geq 1/2$ which gives us our critical probability of $p_c = 1/2$. Applying this same
process to the case where $Z = n \in \mathbb{N}\setminus \{0, 1\}$, it's clear that the bond percolation probability should be $p_c = \frac{1}{Z-1}$.

\subsection{More advanced configurations}
Unfortunately, it's outside the scope of this essay to consider more intricate configurations due to the extreme complexity that arises very quickly. Consider, for example, the
lattice on $\mathbb{Z}^2$. It's troublesome to use the technique of considering clusters and the number of empty sites surrounding such clusters because of the many forms that any
given cluster can take. If we have a cluster, $C$, of size $|C| = 1$ then there's only one possible form for that where it's one open site surrounded by four closed sites giving us a
probability of the site existing of $p(1-p)^4$. Now consider a cluster, $C$, of size $|C|$ = 9. This cluster can take many different shapes that have different "perimiters". It could
be a $3 \times 3$ square of open sites surrounded by $12$ closed ones giving us a probability of $p^9(1-p)^{12}$ of any given site being the start of this cluster. It could also be a $9 \times 1$ line of
open sites that has a total of $20$ closed sites surrounding it resulting in a probability of $p^9(1-p)^{20}$ of any given site being the start of this cluster. The fact that the probability of these
clusters existing isn't uniquely defined by their size makes analysing these problems this way much more problematic. As such, this essay will not explore the methods for
calculating the percolation probabilities for these more complicated configurations.
