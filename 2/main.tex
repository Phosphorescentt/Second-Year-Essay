\section{Analysis of different network configurations}
In this section, we will look at a couple of techniques used to analyse different network configurations to determine their critical probabilities.

\subsection{The one dimensional case}
To develop an understanding of how to analyse these networks, we shall consider the one dimensional case where our network is the lattice on $\mathbb{Z}$, or a "chain".
If we recall the definition of the critical probability, the value of $p_c$ that we're looking for is intuitively $1$ because that's the only way that everything is connected. In
order to more rigorously understand this result, we shall introduce some more
machinery. This machinery isn't necessary for the "proof", but it helps with understanding and will allow us to analyse more interesting cases later on. The following line of
argument is inspired by Dietrich Stauffer's \textit{Introduction to Percolation Theory} \cite{Stauffer}.

If we let each of the sites in our chain be open with probability $p$, then the probability that any $s$ arbitrary sites are all open is $p^s$. Now if we wanted to create a cluster
in our chain of size $s$, we would have to have one closed site followed by $s$ open sites followed by a final closed site to cap it off. This means that the probability of any
given site being the left end of a cluster of size $s$ is $(1-p)p^s(1-p) = p^s(1-p)^2$.

So now how many clusters of size $s$ are there in our chain? If we suppose that the length of the chain is $L >> s$, then we have $Lp^s(1-p)^2$ clusters of size $s$ in our chain of
length $L$ (ignoring small errors produced at the right end of the chain). It's practical for us to talk about the number of clusters at any given site in the chain which is
$\frac{Lp^s(1-p)^2}{L} = p^s(1-p)^2$ which we denote by $n_s$ and refer to as the \textbf{normalised cluster number}. Now it's clear that the probability of any site belonging to a
cluster of size is is larger than our normalised cluster number by a factor of $s$ yielding $n_ss$. Now we can consider the critical probability for this configuration. We want
every site in the chain to be occupied so that there are no holes. If $p = 1$, this is clearly true and the whole chain is one single cluster. For $p < 1$ there will be some holes.
For a chain of length $L$, on average we will have $L(1-p)$ empty sites. As we let $L \rightarrow \infty$ for a fixed $p < 1$ we get that $L(1-p) \rightarrow \infty$. Therefore
there is at least one empty site in the chain so we don't have percolation. Thus $p_c = 1$.

The above results allow us to get some more interesting information about the behaviour of our system. For example, we can deduce the following:
\begin{corollary}
  When considering the lattice on $\mathbb{Z}$ with percolation probability $p \in [0, 1)$, the following equality holds:
  $$\sum_sn_ss = p$$
\end{corollary}
This result comes from the fact that every open site must belong to a cluster of some size $s$. So summing $n_ss$ over all $s$ must give us $p$. This equality may also be derived
using the definition of $n_s$ and the formula for a geometric series.

\begin{proof}
  \begin{align*}
    \sum_sn_ss &= \sum_sp^s(1-p)^2s \\
    &= (1-p)^2\sum_sp\frac{d(p^s)}{dp} \\
    &= (1-p)^2p\frac{d(\sum_sp^s)}{dp} \\
    &= (1-p)^2p\frac{d(p/(1-p))}{dp} \\
    &= p
  \end{align*}
\end{proof}

It's worth noting that this equality doesn't hold for $p=1$, because $n_s=1^s(1-1)^2=0$ so $\sum_sn_ss = 0$.

The technique of considering the sizes of clusters and the number of
empty sites surrounding them is also used when analysing more complex structures. Say, for example, you have a cluster, $C$, of size $|C|=9$ on the lattice on $\mathbb{Z}^2$. This
cluster could take many different shapes with different numbers of empty sites surrounding it. For example, this cluster could be a straight line of open sites which has a total of
$22$ closed sites surrounding it. Therefore the probability that a cluster like this exists at any site given a percolation probability, $p$, is $p^{9}(1-p)^{20}$. This cluster could also be a
square of open sites with side length $3$. This would have $12$ closed sites surrounding it and thus the probability of a cluster of this shape existing at any given site is
$p^{9}(1-p)^{12}$. The ability for clusters to be the same size but have different "perimeters" is what makes analysing these problems in higher dimensions so difficult.
