\documentclass[a4paper,11pt]{article}

\title{\textbf{ A Brief Introduction to Percolation Theory }}
\author{Joshua Mankelow \\ \textit{1902186}}
\date{\today}

% \usepackage{graphicx}
% \usepackage{framed}
% \usepackage{amsmath}
% \usepackage{amssymb}
% \usepackage{amsfonts}
% \usepackage{cite}
% \usepackage{mathrsfs}
% \usepackage{array}
\usepackage{amsthm} %  package used to make the theorem environments work.

% code below sets up new theorem environments
\theoremstyle{plain} % this sets the style for all new environments created using \newtheorem to have the "theorem" style, which as a bold title, italic text and vertical space above and below it. 
\newtheorem{thm}{Theorem}[section] 
\newtheorem{lem}[thm]{Lemma} 
\newtheorem{prop}[thm]{Proposition} 
\newtheorem*{cor}{Corollary} 
\newtheorem*{claim}{Claim} 
\newtheorem*{lagrange}{Lagrange's Theorem} % the star makes an unnumbered theorem

\theoremstyle{definition} % this sets the style for all new environments created using \newtheorem to have the "definition" style, which as a bold title, upright text and vertical space above and below it.
\newtheorem{defn}{Definition}[section]
\newtheorem{eg}{Example}[section]


\theoremstyle{remark} % this sets the style for all new environments created using \newtheorem to have the "remark" style, which as an italic non-bold title, upright text and no extra vertical space above and below it.
\newtheorem*{rem}{Remark} 
\newtheorem{case}{Case}

\begin{document}  
\maketitle

\newpage

\begin{abstract}
  Consider a one metre by one metre cube of water-permeable material. What is the probability that that if water is poured on top of the cube it may percolate all the way through
  the cube and out the opposite face? Initially developed by Paul Flory and Walter Stockmayer in 1944 whilst studying the gelation of polymers, Percolation theory attempts to answer such questions through a statistical analysis of graph configurations in both directed and undirected
  fashions. This essay will introduce the basics of percolation theory, first in one dimension and then working upwards to two dimensions before moving off to talk about some of
  its applications.
\end{abstract}

\newpage

\tableofcontents

\newpage

\section{Introduction}

\end{document}
