\section{Introduction}
\subsection{The canonical example}
Let us consider the example from the abstract of water filtering through a porus medium, but this time in two dimensions. How do we model this? One might imagine that the medium consists of
many particles arranged (for simplicity) in a $n \times n$ square lattice and linked to each of their nearest neighbours. Clearly, this is the lattice on $\mathbb{Z}^2$. To setup the problem,
each of the particles will be expressed as a vertex in a graph and each of the links will be an edge. In the context of percolation, a vertex is called a site and an edge is
called a bond; these edges and bonds form a network. (We're sure the reader can visualise $\mathbb{Z}^2$, but just in case they can't, they may refer to Figure
\ref{fig:squarelattice} beneath) \\

\begin{figure}[h]
  \includegraphics[width=\linewidth]{placeholder}
  \caption{The lattice on $\mathbb{Z}^2$}
  \label{fig:squarelattice}
\end{figure}

So what does percolation actually mean? We shall consider two different types of percolation: bond and site percolation.

\begin{definition}
  We say that we are considering \textbf{site percolation} if we let all of the sites in the network be open with probability $p \in [0, 1]$ (meaning that they allow the liquid
  through) and closed with probability $1-p \in [0, 1]$ (meaning that they don't allow liquid through).
\end{definition}
\begin{definition}
  We say that we are considering \textbf{bond percolation} if we let all of the bonds in the network be open with probability $p \in [0, 1]$ (meaning that they allow the liquid
  through) and closed with probability $1-p \in [0, 1]$ (meaning that they don't allow liquid through).
\end{definition}

Now that we've defined site and bond percolation, what's the problem that we're trying to solve? In the case of water being poured on a porus medium, we would like to know
whether there is a path from the one side of the network to the other. 
