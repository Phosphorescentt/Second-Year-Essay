\section{Introduction}
% \subsection{The canonical example}
Let us consider the example from the abstract of water filtering through a porus medium, but this time in two dimensions. How do we model this? One might imagine that the medium consists of
many particles arranged (for simplicity) in an $n \times n$ square lattice and linked to each of their nearest neighbours. Clearly, this is the lattice on $\mathbb{Z}^2$.
To set up the problem, each of the particles will be expressed as a vertex in a graph and each of the links will be an edge connecting two directly adjacent vertices. In the
context of percolation theory, a vertex is called a site and an edge is
called a bond; these sites and bonds form a graph which we refer to as a network.

% \begin{definition}\label{def:site}
  % A vertex in a graph is referred to as a \textbf{site}.
% \end{definition}

% \begin{definition}\label{def:bond}
  % A edge in a graph is referred to as a \textbf{bond}.
% \end{definition}

% \begin{definition}\label{def:network}
  % A graph is referred to as a \textbf{network}.
% \end{definition}

So what does percolation actually mean? In order to talk about percolation, we need to think about what makes a percolation problem. If we think about the context we're in, some of
the sites in the network will alow the water through and some of them won't. The sites that allow water to pass through are labeled \textbf{open} and the sites that don't allow
water to pass through are labeled \textbf{closed}. This gives us \textbf{site} percolation. If we were to consider open and closed bonds instead of sites, then we would have
\textbf{bond} percolation. We also let these sites or bonds be open with probability $p$ and closed with probability $1-p$. I will refer to this $p$ as the \textbf{percolation
probability}.

% \begin{definition}\label{def:open}
  % A site or bond in a network is labeled \textbf{open} if it allows whatever we're considering to pass through.
% \end{definition}

% \begin{definition}\label{def:closed}
  % A site or bond in a network is labeled \textbf{closed} if it doesn't allow whatever we're considering to pass through.
% \end{definition}

% \begin{definition}\label{def:site percolation}
  % We say that we are considering \textbf{site percolation} if we let all of the sites in the network be open with probability $p \in [0, 1]$ and closed with probability $1-p \in
  % [0, 1]$. We refer to $p$ here as the percolation probability.
% \end{definition}
% \begin{definition}\label{def:bond percolation}
  % We say that we are considering \textbf{bond percolation} if we let all of the bonds in the network be open with probability $p \in [0, 1]$ and closed with probability $1-p \in
  % [0, 1]$. We refer to $p$ here as the percolation probability.
% \end{definition}

Now that we've defined site and bond percolation, what's the problem that we're trying to solve? In the case of water being poured on a porus medium, we would like to know
whether there is a route that the water could take from the top of the medium to the bottom. This is called an \textbf{open path}. We shall model this using site percolation (in fact, all examples in this essay will be using site
percolation unless explicitly stated otherwise).

\begin{definition}\label{def:open path}
  Let $N = (V, E)$ be a network, we say that a path in $N$ is \textbf{open} if every site in the path is open. \footnote{This definition is trivially different for bond
  percolation.}
\end{definition}

\begin{definition}\label{def:openly connected}
  Let $N=(V, E)$ be a network and let $A, B \in V$. The sites $A, B$ are \textbf{openly connected} if there exists an open path connecting $A$ and $B$. Throughout this essay, I
  will interchange the term \textbf{openly connected} with \textbf{connected}.
\end{definition}

\begin{definition}\label{def:openly disconnected}
  Let $N=(V, E)$ be a network and let $A, B \in V$. The sites $A, B$ are \textbf{openly disconnected} if there does not exist an open path connecting $A$ and
  $B$. Throughout this essay, I will intercahnge the term \textbf{openly disconnected} with \textbf{disconnected}. % \footnote{Definitions \ref{def:open path}, \ref{def:openly
  % connected} and \ref{def:openly disconnected} are trivially different for bond percolation.}
\end{definition}

Returning to the example, the probability that an open path from the top of the network to the bottom exists depends on both our choices of both $p$ and $n$. As a result of our context, our value for $n$ should be
large---this is the case with most percolation models---but we shall use small $n$ for the sake of example and simplicity. Let us now fix $n$ and see what happens as we vary $p$. Obviously we have two trivial cases, $p=0$ and $p=1$,
where the network is completely disconnected and completely connected respectively.
What about when $p\in(0,1)$? Let's inspect three different values of $p$ on our network: $p=0.25$, $p=0.5$ and $p=0.75$ as shown in figures \ref{fig:p=0.25}, \ref{fig:p=0.5} and
\ref{fig:p=0.75} on page \pageref{fig:probabilities}.

As one might expect, as $p$ increases, so does the average size of a "cluster" of open sites. Also observe that the overall connectedness of the network increases as $p$ increases.
I.e. the probability of having an open path from the top of the netwoork to the bottom increases with $p$. This is the result that we expect given the context --- as we reduce the
number of things blocking the way for the water, it's easier for it to pass through the block of our chosen medium. Now let us consider two sites $A$ and $B$ in our example
network. As we increase $p$, it's obvious that the probability of these two sites $A$ and $B$ being connected will increse. The question is: What's the relationship between $p$ and
the probability of $A$ and $B$ being connected? It turns out that there's a probability, $p^*$, such that for some $\epsilon$ close to zero
\begin{itemize}
  \item for $p < p^* - \epsilon$, the network is probability of $A$ and $B$ being connected is close to $0$.
  \item for $p > p^* + \epsilon$, the network is probability of $A$ and $B$ being connected is close to $1$.
\end{itemize}

The existence of such a $p^*$ indicates that there must be some threshold where the network transitions from being mostly disconnected to mostly connected. To understand the
structure of the network as a whole and not just two points, we have to consider clusters.

\begin{definition}\label{def:cluster}
  Let $N = (V, E)$ be a network and let $C \subseteq V$. We say that $C$ is a cluster if $\forall u, v \in C$ then $u$ and $v$ are openly connected.
\end{definition}

\begin{definition}\label{def:cluster size}
  Let $N = (V, E)$ be a network and let $C \subseteq V$. We say that $C$ is of size $s$ if $C$ has $s$ vertices. We denote this by $|C| = s$. We say that a cluster $C$ is infinite
  $\iff |C| = \infty$.
\end{definition}

In our example, because we're working in a finite case, there's no single value where this change in structure is obvious --- it happens over a range of values. In the infinite
case however, the change in structure happens instantaneously. I.e. we have the same as the above but without the epsilons. This change in structure is formally defined as the
point at which the network is \textit{guaranteed} to have a cluster of infinite size. The probability $p_c$ that gives rise to this change in structure is referred to as the
\textbf{critical probability} and is dependent on the structure of the network.

\begin{figure}[p]
  \centering
  \begin{subfigure}[b]{0.45\textwidth}
    \centering
    \includegraphics[width=\textwidth]{1/percolation1}
    \caption{$p=0.25$}
    \label{fig:p=0.25}
  \end{subfigure}
  \hfill
  \begin{subfigure}[b]{0.45\textwidth}
    \centering
    \includegraphics[width=\textwidth]{1/percolation2}
    \caption{$p=0.5$}
    \label{fig:p=0.5}
  \end{subfigure}
  \hfill
  \begin{subfigure}[b]{0.45\textwidth}
    \centering
    \includegraphics[width=\textwidth]{1/percolation3}
    \caption{$p=0.75$}
    \label{fig:p=0.75}
  \end{subfigure}
  \caption{Examples of bond percolation for $p\in(0,1)$ on a $40 \times 40$ network where if a site is open in the model it appears present in the diagram and bonds are only
  present if they connect two open sites.}
  \label{fig:probabilities}
\end{figure}

\subsection{Other network configurations}
We have already seen the lattice on $\mathbb{Z}^2$ as an example of one network configuration, but there are many more. To remain within the scope of this essay, we shall only breifly mention some two and
three dimensional examples and print their site and bond critical probabilities and a diagram. Before talking about different network configurations, it's important to introduce the
following two qualities that we use to describe them.

\begin{itemize}
  \item A network is considered \textbf{regular} if every site in that network has the same number of bonds attached to it.
  \item The \textbf{coordination number} of a regular network is the number of bonds attached at every site. This quantity is denoted using the letter $Z$. I.e. the lattice on
    $\mathbb{Z}^2$ has a coordination number of $Z = 4$.
\end{itemize}

% \begin{definition}
  % A network is considered \textbf{regular} if every site in that network has the same number of bonds attached to it.
% \end{definition}

% \begin{definition}
  % The \textbf{coordination number} of a regular network is the number of bonds attached at every site. This quantity is denoted using the letter $Z$. I.e. the lattice on
  % $\mathbb{Z}^2$ has a coordination number $Z = 4$. 
% \end{definition}

\subsection*{Two dimensional network configurations}
Clearly, one two dimensional network configuration is the lattice on $\mathbb{Z}^2$. In context this is referred to as the square lattice. Other regular two dimensional network configurations include, but are not limited to, the Bethe Lattice (Figure \ref{fig:bethe lattice}), Honeycomb
Lattice (Figure \ref{fig:honeycomb lattice}), Kagome Lattice (Figure \ref{fig:kagome lattice}) and the Triangular Lattice (Figure \ref{fig:triangular lattice}). As one might
imagine, each of these configurations has a different (but not necessarily distinct) critical probability. Below is a table showing the critical probabilities for each of the
aforementioned network configurations. It should be noted that probabilities marked with a * (star) are exact results.

\begin{figure}[h!]
\begin{center}
\begin{tabular}{| c | c | c | c |}
    \hline
    Configuration & Z & $p_c$ for bond percolation & $p_c$ for site percolation \\
    \hline
    Bethe ($Z=3$) & $3$ & $0.5$ & $1/3$ \\
    Honeycomb & $3$ & $1 - 2\sin(\pi/18)$* & $0.6962$ \\
    $\mathbb{Z}^2$ (Square) & $4$ & $1/2$* & $0.5927$ \\
    Kagome & $4$ & $0.522$ & $0.652$ \\
    Triangular & $6$ & $2\sin(\pi/18)$* & $1/2$* \\
    \hline
  \end{tabular}
\end{center}
\centering
\caption{Critical probabilities for various configurations of two dimensional networks\cite[p. 11]{Sahimi}}
\label{fig:critical probabilities in two dimensions}
\end{figure}

\begin{figure}[h!]
\begin{center}
\begin{tabular}{| c | c | c | c |}
    \hline
    Configuration & Z & $p_c$ for site percolation & $p_c$ for bond percolation \\
    \hline
    Diamond & $4$ & $0.3886$ & $0.4299$ \\
    Simple Cubic & $6$ & $0.2488$ & $0.3116$ \\
    BCC & $8$ & $0.1795$ & $0.2464$ \\
    FCC & $12$ & $0.198$ & $0.119$ \\
    \hline
  \end{tabular}
\end{center}
\centering
\caption{Critical probabilities for various configurations of three dimensional networks\cite[p. 11]{Sahimi}}
\label{fig:critical probabilities in three dimensions}
\end{figure}

\begin{figure}[p]
  \centering
  \begin{subfigure}[b]{0.45\textwidth}
    \centering
    \includegraphics[width=\textwidth]{2/Bethe}
    \caption{Bethe Lattice}
    \label{fig:bethe lattice}
  \end{subfigure}
  \hfill
  \begin{subfigure}[b]{0.45\textwidth}
    \centering
    \includegraphics[width=\textwidth]{2/Honeycomb}
    \caption{Honeycomb Lattice}
    \label{fig:honeycomb lattice}
  \end{subfigure}
  \hfill
  \begin{subfigure}[b]{0.45\textwidth}
    \centering
    \includegraphics[width=\textwidth]{2/Kagome}
    \caption{Kagome Lattice}
    \label{fig:kagome lattice}
  \end{subfigure}
  \hfill
  \begin{subfigure}[b]{0.45\textwidth}
    \centering
    \includegraphics[width=\textwidth]{2/Triangular}
    \caption{Triangular Lattice}
    \label{fig:triangular lattice}
  \end{subfigure}
  \caption{Two dimensional regular network configurations}
  \label{fig:two dimensional networks}
\end{figure}

\subsection*{Three dimensional network configurations}
It shouldn't be hard to guess that the lattice on $\mathbb{Z}^3$ is a potential configuration for three dimensional networks. We call this configuration the Simple Cubic Lattice (Figure
\ref{fig:simple cubic lattice}). Similarly to
the two dimensional case, there many other regular three dimensional network configurations. These include, but again are not limited to, the Diamond Lattice (Figure
\ref{fig:diamond lattice}), the Body Centered Cubic (BCC) Lattice (Figure \ref{fig:body centered cubic lattice}) and the Face Centered Cubic (FCC) Lattice (Figure \ref{fig:face
centered cubic lattice}). Notice how none of these results are precise.

\begin{figure}[p]
  \centering
  \begin{subfigure}[b]{0.45\textwidth}
    \centering
    \includegraphics[width=\textwidth]{images/placeholder}
    \caption{Diamond Lattice}
    \label{fig:diamond lattice}
  \end{subfigure}
  \hfill
  \begin{subfigure}[b]{0.45\textwidth}
    \centering
    \includegraphics[width=\textwidth]{images/placeholder}
    \caption{Simple Cubic Lattice}
    \label{fig:simple cubic lattice}
  \end{subfigure}
  \hfill
  \begin{subfigure}[b]{0.45\textwidth}
    \centering
    \includegraphics[width=\textwidth]{images/placeholder}
    \caption{Body Centered Cubic Lattice}
    \label{fig:body centered cubic lattice}
  \end{subfigure}
  \hfill
  \begin{subfigure}[b]{0.45\textwidth}
    \centering
    \includegraphics[width=\textwidth]{images/placeholder}
    \caption{Face Centered Cubic Lattice}
    \label{fig:face centered cubic lattice}
  \end{subfigure}
  \caption{Three dimensional regular network configurations}
  \label{fig:three dimensional networks}
\end{figure}
